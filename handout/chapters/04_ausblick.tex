\section{Ausblick}

\subsection{Aktuelle Schwächen}

Die Integration von \ac{llm} in den Bereich der statischen Programmanalyse bietet viel Potential, steht jedoch erst am Anfang Ihrer Entwicklung. Aktuelle Forschungsarbeiten in dem Bereich zeigen neben dem Potential auch Limitierungen auf.

Viele Ansätze in dem Gebiet beziehen sich auf konkrete Programmiersprachen und Benchmarks. Während erste Studien auch ungelabelte Daten in Ihre Versuche mit einbeziehen\cite{wagnerEffectiveComplementarySecurity2025}, ist das Feld noch nicht flächendeckend erforscht.  Die Performance von \ac{llm} ist stark abhängig vom bereitgestellten Kontext und dem Prompting\cite{wagnerEffectiveComplementarySecurity2025}.  Deshalb ist es trotz justierter Parameter wie der temperature möglich,  nicht-deterministische Ergebnisse auszuschließen\cite{khareUnderstandingEffectivenessLarge2024}.

Eine weitere Dimension des Einsatzes von \ac{llm} sind die entstandenen Kosten. Bei jedem Aufruf des \ac{llm}, z.B. bei der Spezifikationsgenerierung oder Ergebnisfilterung entstehen Kosten. Diese Skalieren mit der Codebasis und Analysekomplexität\cite{liEnhancingStaticAnalysis2024}. 

\subsection{Zukünftige Forschung}
Die Verbindung zwischen dem Analysetool und dem \ac{llm} sollte weiter untersucht und verbessert werden\cite{khareUnderstandingEffectivenessLarge2024}. Zudem sollte weitergehend Untersucht werden, wie die Interaktion mit dem Sprachmodell optimiert werden können. Dabei können verschiedene Parameter untersucht werden, unter anderem die Parametergröße des Modells, um Kosten/Zeit und Effektivität zu maximieren. General Purpose Modelle zeigen bereits vielversprechende Ergebnisse. Dennoch sollten auch Fine-Tuned \acp{llm} sollten im Feld der statischen Analyse untersucht werden\cite{wagnerEffectiveComplementarySecurity2025}.
