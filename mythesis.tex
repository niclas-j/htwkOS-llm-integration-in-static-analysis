% **************************************************
% Document Class Definition
% **************************************************
\documentclass[%
    paper=A4,               % paper size --> A4 is default in Germany
    ngerman,
    %twoside=true,           % onesite or twoside printing
    %openright,              % doublepage cleaning ends up right side
    parskip=half,           % spacing value / method for paragraphs
    %chapterprefix=true,     % prefix for chapter marks
    11pt,                   % font size
    headings=normal,        % size of headings
    bibliography=totoc,     % include bib in toc
    listof=totoc,           % include listof entries in toc
    %titlepage=on,           % own page for each title page
    %captions=tableabove,    % display table captions above the float env
    chapterprefix=false,    % do not display a prefix for chapters
    appendixprefix=false,    % but display a prefix for appendix chapter
    draft=false,            % value for draft version
]{scrartcl}%

% **************************************************
% Setup YOUR thesis document in this file !
% **************************************************
\input{my-thesis-setup}



\title{Verbesserte statische Programm-Analyse durch die Integration von LLM}
\author{Niclas Jost}
\date{}
% **************************************************
% Document CONTENT
% **************************************************
\begin{document}


\maketitle
\thispagestyle{scrheadings}
% **************************************************
% Main CONTENT
% **************************************************
% **************************************************
% **************************************************
\section{Vision} 

%Was ist die Vision hinter dem Thema? Wie ordnet sich
%diese zu anderen Themen und grundsätzlichen Fragen
% ein?

Moderne Programme kommen mit inhärenter Komplexität einher. Es bestehen Abhängigkeiten zwischen Bibliotheken, Modulen und externem Code \cite{khareUnderstandingEffectivenessLarge2024}. Dadurch herrscht ein hoher Bedarf an zuverlässigen Tools, um einen hohen Qualitätsstandard zu gewährleisten.

\acp{llm} haben in den letzten Jahren Fortschritte in Aufgabenbereichen wie Program-Repair \cite{xiaAutomatedProgramRepair2023} und Code/Test-Generierung \cite{lemieuxCodaMosaEscapingCoverage2023} erreicht. Auch die statische Programmanalyse kann davon profitieren. Die gezielte Integration von \acp{llm} in den Workflow der statischen Analyse ermöglicht es, die jeweiligen Stärken komplementär zu nutzen \cite{liEnhancingStaticAnalysis2024}\cite{wagnerEffectiveComplementarySecurity2025}.
Die Fähigkeit von \acp{llm}, ein intuitives und menschenähnliches Verständnis von Code zu zeigen, stellt eine Abgrenzung und Ergänzung zu den formalen, regelbasierten Ansätzen dar, welche die traditionelle statische Analyse charakterisieren\cite{liEnhancingStaticAnalysis2024}.



\selectlanguage{\ngerman}


\section{Potenziale und Ziele}\label{sec:potent}

Die Integration von \acp{llm} in die statische Programmanalyse hat das Potenzial, die Schwächen der traditionellen Ansätze zu reduzieren. Der Fokus liegt hierbei nicht auf der Ersetzung bestehender Verfahren, sondern deren Effektivität durch die Kombination mit \acp{llm} zu steigern\cite{khareUnderstandingEffectivenessLarge2024}. Kernbereiche dieser Unterstützung umfassen die direkte Verbesserung der Analyseergebnisse sowie der Kompensation von Wissensdefiziten konventioneller Werkzeuge.

\subsection{Verbesserung der Analyseergebnisse}

Die Ausgabe statischer Analysewerkzeuge neigt dazu, eine hohe Rate an \acfp{fp} zu generieren. Das ist ein großes Problem. Jedes \ac{fp} erfordert eine manuelle Überprüfung und Bindung von Entwicklungsressourcen. Zudem sinkt das Vertrauen in die Werkzeuge durch eine niedrige Precision\cite{wagnerEffectiveComplementarySecurity2025}.

Der gezielte Einsatz von \acp{llm} kann die Rate an \ac{fp} die die Entwickler erreicht reduzieren\cite{wagnerEffectiveComplementarySecurity2025}. Auch die Identifikation komplett neuer Probleme bei denen traditionelle Tools an ihre Grenzen stoßen ist möglich. In großen Codebasen müssen konventionelle Tools einen Trade-Off zwischen Skalierbarkeit und Präzision eingehen.

\begin{itemize}
    \item Präzise Analyse: Konzepte wie path-sensitive Analyse bringen eine hohe Präzision, scheitern aber in komplexen Programmen. So erreicht das Tool UBITECT im Linux Kernel bei ca. 40Prozent der potenziellen Bugs ein Speicherlimit\cite{liEnhancingStaticAnalysis2024}
\item Skalierbare Analyse: Nutzen gröberes Vorgehen um auch komplexe Programme Rechen und Zeiteffizient zu analysieren. Diese leiden aber unter vielen \ac{fp} 
\end{itemize} 

\ac{llm} bieten das Potenzial, durch eine gezielte Integration in der Analysekette die Analyseergebnisse positiv zu beeinflussen\cite{chapmanInterleavingStaticAnalysis2024}.

\subsection{Wissenslücken schließen}

Analysetools können kein vollständiges Wissen über alle Funktionen, Bibliotheken, Sprachmerkmale oder Laufzeitumgebungen eines komplexen Softwaresystems besitzen \cite{liEnhancingStaticAnalysis2024}. Konventionelle Tools sind dafür abhängig von menschlicher Modellierung basierend auf deren Wissen. Dieser Vorgang ist fehleranfällig und kostet Arbeitskraft\cite{liIRISLLMAssistedStatic2024}. Sprachmodelle können durch Trainingsdaten und kontextuelles Verständnis diese Limitation begrenzen und dem statischen Analysetool wichtige Daten bereitstellen\cite{liEnhancingStaticAnalysis2024}. 


\section{Umsetzung}

Es existieren verschiedene Integrationspunkte für das \ac{llm} in die statische Programmanalyse. Dies erfolgt nicht monolithisch, sondern durch eine Integration des \ac{llm} an verschiedenen Punkten im Analyseprozess.


\subsection{Architektonische Muster der LLM-Integration in die statische Analyse}
\begin{table}[ht]
\centering
\begin{tabularx}{\textwidth}{>{\raggedright\arraybackslash}X >{\raggedright\arraybackslash}X >{\raggedright\arraybackslash}X}
\hline
\textbf{Muster} & \textbf{Primäres Ziel} & \textbf{Haupteinfluss auf Metriken} \\
\hline
Pre-Processing (Spezifikationsgenerierung) & Automatisierte Spezifikationserstellung; Reduktion des manuellen Aufwands & $\uparrow$ Recall, $\uparrow$ Präzision \\
\hline
Post-Processing (Ergebnisfilterung) & Reduktion von False Positives (FP) bei Beibehaltung der True-Positive-Rate (TPR) & $\uparrow$ Präzision, TPR bleibt idealerweise konstant \\
\hline
Interleaved / Synergistic (Neuro-Symbolisch) & Lösung von Analyseproblemen, die für traditionelle Tools zu komplex sind (z.B. durch Pfadexplosion) & $\uparrow$ Recall (durch Erkennung neuer Fehlerklassen) \\
\hline
\end{tabularx}
\caption{Vergleich der LLM-Integrationsmuster in der statischen Analyse.}
\label{tab:vergleich_ansaetze}
\end{table}

\subsubsection{Pre-Processor - Spezifikationsgenerierung}
Das \ac{llm} kann genutzt werden, um notwendige, aber oft fehlende Spezifikationen die zur eigentlichen konventionellen statischen Analyse notwendig sind zu generieren Dieser Ansatz behebt eine zentrale Schwäche der statischen Taint-Analyse: deren Abhängigkeit von manuellem Labeling von Sources, Sinks und Sanitizern \cite{liIRISLLMAssistedStatic2024}.

Das Framework  IRIS demonstriert diesen Ansatz (Abb \ref{fig:pre_processing_pattern}, indem es ein \ac{llm} nutzt, um API-Kandidaten automatisiert zu klassifizieren und die resultierenden Spezifikationen an ein Analysewerkzeug wie CodeQL zu übergeben.\cite{liIRISLLMAssistedStatic2024}.

\begin{figure}[h]
\centering
% Option 3: Kompaktes, zweizeiliges Layout
\begin{tikzpicture}[
    node distance = 1.5cm and 3cm, % vertikaler und horizontaler Abstand
    block/.style = {rectangle, draw, thick, text width=4cm, minimum height=1.5cm, text centered, font=\sffamily},
    arrow/.style = {-{Stealth[length=3mm]}}
]

% Nodes anordnen
\node[block] (code) {Projekt-Quellcode};
\node[block, below=of code] (extractor) {1. Candidate Extractor (SA)};
\node[block, right=of extractor] (llm) {2. LLM zur Klassifizierung};

% Zweite Zeile
\node[block, below=of llm] (taint) {3. Statische Taint-Analyse (z.B. CodeQL)};
\node[block, below=of taint] (result) {Analyseergebnisse (Gefundene Schwachstellen)};

% Pfeile
\draw[arrow] (code) -- (extractor);
\draw[arrow] (extractor) -- node[midway, above] {API-Kandidaten} (llm);
\draw[arrow] (llm) -- node[midway, right] {Generierte Spezifikationen} (taint);
\draw[arrow] (taint) -- (result);

% Gestrichelter Pfeil vom Quellcode
\draw[arrow, dashed] (code.east) to[bend left=20] (taint.north);

\end{tikzpicture}
\caption{Vereinfachtes Pre-Processing (nach IRIS \cite{liIRISLLMAssistedStatic2024})}
\label{fig:pre_processing_pattern}
\end{figure}

\subsubsection{Post-Processor - Verarbeitung von Ergebnissen}
Im Post-Processing-Ansatz wird das Ergebnis eines traditionellen Analysetools durch ein nachgeschaltetes \ac{llm} verfeinert\cite{wagnerEffectiveComplementarySecurity2025}. Das Hauptziel dieser Integration liegt in der Reduktion der \ac{fp}-Meldungen. Dabei ist ein Aufrechterhalten der \ac{TPR} essentiell, um sicherzustellen, dass keine Schwachstellen als harmlos eingestuft werden\cite{wagnerEffectiveComplementarySecurity2025}.

\begin{figure}
\centering
\begin{tikzpicture}[
    node distance = 2cm,
    block/.style = {
        rectangle, 
        draw, 
        thick,
        text width=3cm,
        minimum height=1.5cm, 
        text centered,
        font=\sffamily
    },
    arrow/.style = {-{Stealth[length=3mm]}}
]

\node[block] (sa) {Statische Analyse};
\node[block, right=of sa] (llm) {LLM-basierte Filterung};
\node[block, right=of llm] (result) {Gefilterte Ergebnisse};

\draw[arrow] (sa) -- (llm);
\draw[arrow] (llm) -- (result);

\end{tikzpicture}
\caption{Vereinfachtes Post-Processing}
\label{fig:pre_processing_pattern}
\end{figure}

\subsubsection{Interleaved Systeme}
Ein fortschrittliches Muster, bei dem die konventionelle statische Analyse im zusammenspiel mit der \ac{llm} Integration arbeitet. Das ermöglicht es, Blockaden zu überwinden, die für rein formale Methoden zu komplex sind.

\textbf{Ansatz als Fallback-Mechanismus (LLIFT):} Scheitert die präzise Analyse eines Pfades an dessen Komplexität (z.B. Timeout durch Pfadexplosion), wird das LLM als intelligenter Assistent herangezogen, um eine menschenähnliche Analyse des spezifischen, schwer entscheidbaren Falles durchzuführen\cite{khareUnderstandingEffectivenessLarge2024}.

\begin{figure}
\centering
\begin{tikzpicture}[
    % Die Abstände können wir beibehalten oder leicht anpassen
    node distance = 1.5cm and 2cm,
    block/.style = {
        rectangle, 
        draw, 
        thick,
        text width=2.5cm,
        minimum height=1.5cm, 
        text centered,
        font=\sffamily
    },
    arrow/.style = {-{Stealth[length=3mm]}}
]

\node[block] (sa) {Statische Analyse};
\node[block, right=of sa] (se) {Symbolic Execution};
\node[block, right=of se] (result) {Result};
\node[block, below=of result] (llm) {LLM Post-processing};
\draw[arrow] (sa) -- (se);
\draw[arrow] (se) -- node[above, font=\sffamily\small] {decided} (result);

\draw[arrow] (se.south) to[bend right=45] node[below, pos=0.4, font=\sffamily\small] {undecided cases} (llm.west);

\draw[arrow] (llm) -- (result);

\end{tikzpicture}
\caption{Eingriff in Analyse(nach LLIFT\cite{khareUnderstandingEffectivenessLarge2024})}
\label{fig:pre_processing_pattern}
\end{figure}

\textbf{Ansatz als iterative Schleife (EESI):} Die statische Analyse übergibt ihre \textit{Zwischenergebnisse} (z.B. bereits bekannte Eigenschaften von aufgerufenen Funktionen) als hochrelevanten Kontext an das LLM. Die Antwort des LLMs wird wiederum als neue Erkenntnis in die laufende statische Analyse zurückgespeist, um deren Präzision zu erhöhen\cite{chapmanInterleavingStaticAnalysis2024}.


\subsection{Prompting Strategien}

Alle Muster benötigen eine bedachte Prompting-Strategie, um gute Ergebnisse zu erzielen. Dazu gehören unter anderem:

\begin{itemize}

\item Chain-of-Thought: Bringt das Modell dazu, schrittweise zu denken, um Resultate bei komplexen Aufgaben zu verbessern \cite{liEnhancingStaticAnalysis2024}.
\item Few-Shot-Prompting: Hinzufügen von Beispielen zu einer Prompt welche das  Ergebnis durch \ac{icl} verbessern \cite{khareUnderstandingEffectivenessLarge2024}.
\item Progressive Prompt: Erlaubt dem LLM zusätzliche Kontextrelevante Informationen bei Bedarf abzufragen, um Token-Einschränkungen zu begrenzen \cite{liEnhancingStaticAnalysis2024}.

\end{itemize}

Modellparameter wie $max_tokens$ und $temperature$ haben einen entscheidenden Einfluss auf die Konsistenz und Qualität der Ergebnisse \cite{wagnerEffectiveComplementarySecurity2025}.


\section{Ausblick}

\subsection{Aktuelle Schwächen}

Die Integration von \ac{llm} in den Bereich der statischen Programmanalyse bietet viel Potential, steht jedoch erst am Anfang Ihrer Entwicklung. Aktuelle Forschungsarbeiten in dem Bereich zeigen neben dem Potential auch Limitierungen auf.

\textbf{Fehlender Gesamtkontext:} Die Analyse bezieht sich nur auf bereitgestellte Code-Ausschnitte. Dadurch können Kontextlücken entstehen, welche das Ergebnis beeinflussen\cite{liIRISLLMAssistedStatic2024}.

\textbf{Determinismus} 

\textbf{Betriebskosten}

\textbf{Studiendesign und Bias}

\subsection{Zukünftige Forschung}

\textbf{Vertiefung der neuro-symbolischen Integration}

\textbf{Optimierung Modellinteraktion (Multiple Modelle, Prompting, Parametergröße}



% **************************************************
% **************************************************
% **************************************************
% Back matter
% **************************************************
%
{%
\setstretch{1.1}
\renewcommand{\bibfont}{\normalfont\small}
\setlength{\biblabelsep}{0.25em}
\setlength{\bibitemsep}{0.5\baselineskip plus 0.5\baselineskip}
% print all references that are not ot type online
\printbibliography[nottype=online]
\newrefcontext[labelprefix={@}]
% web pages are typically non-scientific resources, so we separate them from the scientific ones
% typically 
\printbibliography[heading=subbibliography,title={Webpages},type=online]
}

\listoffigures
\thispagestyle{scrheadings}
\listoftables
\thispagestyle{scrheadings}
\begin{acronym}[ECU]
\acro{llm}[LLM]{Large Language Model}
\acro{icl}[ICL]{In-Context-Learning}
\acroplural{llm}[LLMs]{Large Language Models}
\acro{tp}[TP]{True-Positive}
\acro{fp}[FP]{False-Positive}
\acroplural{fp}[FP]{False-Positives}
\acro{fn}[FN]{False-Negative}
\acro{tpr}[TPR]{True-Positive-Rate}
\end{acronym}
%\lstlistoflistings



%\appendix
%\input{content/appendix}       % INCLUDE: appendix

% **************************************************
% End of Document CONTENT
% **************************************************
\end{document}
